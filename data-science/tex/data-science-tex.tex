% Created 2018-04-15 Sun 17:34
\documentclass[11pt]{article}
\usepackage[utf8]{inputenc}
\usepackage[T1]{fontenc}
\usepackage{fixltx2e}
\usepackage{graphicx}
\usepackage{longtable}
\usepackage{float}
\usepackage{wrapfig}
\usepackage{rotating}
\usepackage[normalem]{ulem}
\usepackage{amsmath}
\usepackage{textcomp}
\usepackage{marvosym}
\usepackage{wasysym}
\usepackage{amssymb}
\usepackage{hyperref}
\tolerance=1000
\usepackage{amsmath}
\usepackage{paralist}
\usepackage[utf8]{inputenc}
\usepackage{palatino}
\usepackage{euler}
\usepackage{setspace}
\renewcommand{\em}[1]{\textbf{#1}}
\newcommand{\E}[1]{\operatorname{\mathbb{E}}[#1]}
\setstretch{1.2}
\let\itemize\compactitem
\let\description\compactdesc
\let\enumerate\compactenum
\setlength{\parindent}{0em}
\setlength{\parskip}{1em}
\newcommand{\RR}{\mathbb{R}}
\author{Bart Frenk}
\date{\today}
\title{In-depth notes on data science related topics}
\hypersetup{
  pdfkeywords={},
  pdfsubject={},
  pdfcreator={Emacs 25.1.1 (Org mode 8.2.10)}}
\begin{document}

\maketitle

\section{Topics}
\label{sec-1}
\subsection{Generalized linear models}
\label{sec-1-1}
\subsubsection{Introduction}
\label{sec-1-1-1}

Ingredients:

\begin{itemize}
\item a \em{dependent variable} $Y$ that follows a particular distribution in the exponential family,
\item a vector of \em{independent variables} $X$,
\item an invertible function $g: \mathbb{R} \rightarrow \mathbb{R}$, referred to as the \em{link function},
\item an \em{unknown vector of coefficients} $\beta$, of the same length as $X$
\end{itemize}

The following relation is assumed to hold:

\begin{equation}
g^{-1}(\E{Y|X}) = \beta \cdot X.
\end{equation}

\subsubsection{Examples}
\label{sec-1-1-2}

\subsubsection{Bayesian regression}
\label{sec-1-1-3}
\subsection{Stochastic approximation}
\label{sec-1-2}
This is important for both online learning (e.g., iteratively computing maximum
likelihood estimators as the data comes in, as well as for dealing with large
data sets).
\subsubsection{Sequential estimators and the Robbins-Monro algorithm}
\label{sec-1-2-1}
The original article is \footnote{Robbins, Monro. A stochastic approximation method (1951)}. It deals solely with the case of stochastically
approximating solutions of equations of the form $M(x) = \alpha$, in which
$\alpha$ is a fixed constant, and $M(x)$ is the conditional expectation of a
random variable $Y$ given $X = x$, i.e.,
\begin{equation}
M(x) = \E{Y \mid X=x}
\end{equation}
The function $M$ need not be completely specified, but there should be a way of sampling
from $Y$ given $X=x$.

In section 2.3.5 of \footnote{Christopher M. Bishop. Pattern recognition and machine learning (2009)} there is a method to derive a sequential method for
computing maximum likelihood estimator using the method of Robbins and Monro.

There is a good exposition of this method in the introduction of \footnote{Toulis ea - Stable Robbins-Monro approximations through stochastic proximal updates (2018)}.

From the article:

Consider the problem of estimating the zero $\theta_*$ of a function $h:
\mathbb{R}^p -> \mathbb{R}$, where $(\theta)$ is unknown bu can be unbiasedly
estimated by a random variable $W_{\theta}$ such that $\E{W_{\theta}} =
h(\theta)$. Starting from an estimate $\theta_0$, Robbins and Munro iteratively
estimated $\theta_*$ as follows:
\begin{equation}
\theta_n = \theta_{n - 1} - \gamma_n W_{\theta_{n - 1}},
\end{equation}
where $(\theta_n)$ is usually a decreasing sequence of positive numbers, known
as the \em{learning rate sequence}. Typically, we choose $\gamma_n \propto 1 /
n$, for $n = 1, 2, \ldots$, so that $\sum \gamma_i^2 < \infty$ and $\sum
\gamma_i = \infty$.

\subsubsection{Stochastic gradient descent}
\label{sec-1-2-2}
This should be an example of stochastic approximation. Does it fit in the
Robbins-Monro framework?







\subsubsection{Multi-armed bandits}
\label{sec-1-2-3}
\begin{enumerate}
\item Simplest formulation
\label{sec-1-2-3-1}
Ingredients:

\begin{itemize}
\item observation $Y$ in some space $\mathcal{Y}$.
\item reward function $r: \mathcal{Y} \rightarrow \RR$:
\item set of actions $\mathcal{A}$.
\end{itemize}
\item Separate reward from observations
\label{sec-1-2-3-2}
\item With context
\label{sec-1-2-3-3}
\item Time dependence of the parameters
\label{sec-1-2-3-4}
\end{enumerate}
% Emacs 25.1.1 (Org mode 8.2.10)
\end{document}